
\documentclass{article}
\usepackage{color}
\usepackage{minted}


\begin{document}
\title{SQL Course}
\author{Konrad Kadzielawa}
\date{March 2023}
\maketitle


\section*{Chapter 1: Machine Learning Landscape}

\section{Types of Machine Learning Systems}
Useful to break down ML system types into 3 categories:

\begin{itemize}	
	\item based on supervision (labeling)
		\begin{itemize}
			\item supervised learning
			\item unsupervised learning
			\item semisupervised learning
			\item reinforcement learning
		\end{itemize}		

	\item based on incremental learning 
		\begin{itemize}
			\item online learning
			\item batch learning
		\end{itemize}		

	\item based on learning type
		\begin{itemize}
			\item instance-based
			\item model-based
		\end{itemize}		
\end{itemize}


\subsection{Supervised/Unsupervised Learning}
Classify according to amount and type of supervision during training.
\subsubsection{Supervised Learning}
Training data includes the desired solution called \textbf{labels}.
\begin{minted}{python}
import numpy as np
    
def incmatrix(genl1,genl2):
    m = len(genl1)
    n = len(genl2)
    M = None #to become the incidence matrix
    VT = np.zeros((n*m,1), int)  #dummy variable
    
    #compute the bitwise xor matrix
    M1 = bitxormatrix(genl1)
    M2 = np.triu(bitxormatrix(genl2),1) 

    for i in range(m-1):
        for j in range(i+1, m):
            [r,c] = np.where(M2 == M1[i,j])
            for k in range(len(r)):
                VT[(i)*n + r[k]] = 1;
                VT[(i)*n + c[k]] = 1;
                VT[(j)*n + r[k]] = 1;
                VT[(j)*n + c[k]] = 1;
                
                if M is None:
                    M = np.copy(VT)
                else:
                    M = np.concatenate((M, VT), 1)
                
                VT = np.zeros((n*m,1), int)
    
    return M
\end{minted}

\begin{minted}{postgresql}



select * 
from employee_info
group by 1,2
having count(1) > 0
;



\end{minted}









\end{document}